\documentclass[11pt,a4paper]{article}
\usepackage[utf8]{inputenc}
\usepackage[spanish]{babel}
\usepackage{hyperref}

\title{Análisis de Seguridad en la Nube: Informe y Caso Práctico}
\author{Equipo de Práctica}
\date{\today}

\begin{document}
\maketitle
\begin{abstract}
Resumen del informe.
\end{abstract}

\section*{1. Resumen}
Contexto general, problema abordado, metodología, resultados.

\section*{2. Introducción}
Importancia de la computación en la nube, crecimiento, riesgos, justificación.

\section*{3. Objetivo principal}
Analizar la seguridad en la nube mediante un enfoque teórico y un caso práctico.

\section*{4. Objetivos específicos}
Identificar principios, analizar responsabilidad compartida, evaluar controles, determinar riesgos y buenas prácticas.

\section*{5. Marco teórico}
Definiciones: IaaS, PaaS, SaaS; CIA; modelo de responsabilidad compartida; IAM; seguridad de red; seguridad de aplicaciones; normativas.

\section*{6. Desarrollo}
\subsection*{Entorno}
Descripción del entorno utilizado (VM, Docker, servicios GCP si aplica).
\subsection*{Arquitectura}
Diagrama y explicación.
\subsection*{Configuración}
Controles aplicados, IAM, firewall, KMS, logging.
\subsection*{Análisis del ejemplo práctico}
Descripción del experimento, comandos, resultados.

\subsection*{Resultados obtenidos}
Se desplegó la aplicación en una VM con Docker y se verificaron los endpoints y el contenedor. A continuación se muestran los comandos ejecutados y sus salidas (salida recabada desde la VM):

\begin{verbatim}
# Peticiones a los endpoints (desde la VM)
curl -s http://127.0.0.1:8080/
{"message":"CloudSecurity example app","host":"01945135b18e"}

curl -s http://127.0.0.1:8080/health
{"status":"ok"}

curl -s -H "x-api-key: clave_demo_segura" http://127.0.0.1:8080/secure
{"secret":"datos-sensibles-de-ejemplo"}

# Estado de contenedores
docker ps
CONTAINER ID   IMAGE                           COMMAND                  CREATED         STATUS         PORTS                                         NAMES
01945135b18e   cloudsec:latest                 "docker-entrypoint.s..."   5 minutes ago   Up 5 minutes   0.0.0.0:8080->3000/tcp, [::]:8080->3000/tcp   cloudsec
c24c8508bdea   frontend-leccion1-3u-frontend   "docker-entrypoint.s..."   5 days ago      Up 3 days      0.0.0.0:5173->5173/tcp, [::]:5173->5173/tcp   leccion3u-frontend

# Logs del contenedor (ejemplo)
docker logs -f cloudsec
App listening on port 3000
\end{verbatim}

Observaciones:
- El contenedor `cloudsec` está mapeado al puerto 8080 en la VM y responde correctamente a las peticiones.
- Espacio disponible en raíz antes del despliegue: aproximadamente 3.1GB. Se recomienda limpieza de imágenes con `docker image prune -f` si se requiere liberar espacio.
- Para pruebas remotas desde Internet fue posible acceder a la IP pública del host en el puerto 8080 (ejemplo mostrado en navegador). Si no se puede acceder desde fuera, revisar reglas de firewall en la plataforma cloud.

\section*{7. Discusión}
Evaluación, comparación teoría/práctica, riesgos identificados y limitaciones.

\section*{8. Conclusiones}
Cumplimiento de objetivos y aportes.

\section*{9. Recomendaciones}
Buenas prácticas y mejoras.

\section*{10. Fuentes bibliográficas}
Lista de referencias.

\end{document}
