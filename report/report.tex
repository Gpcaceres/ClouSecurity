\documentclass[11pt,a4paper]{article}
\usepackage[utf8]{inputenc}
\usepackage[spanish]{babel}
\usepackage{hyperref}

\title{Análisis de Seguridad en la Nube: Informe y Caso Práctico}
\author{Equipo de Práctica}
\date{\today}

\begin{document}
\maketitle
\begin{abstract}
Resumen del informe.
\end{abstract}

\section*{1. Resumen}
Contexto general, problema abordado, metodología, resultados.

\section*{2. Introducción}
Importancia de la computación en la nube, crecimiento, riesgos, justificación.

\section*{3. Objetivo principal}
Analizar la seguridad en la nube mediante un enfoque teórico y un caso práctico.

\section*{4. Objetivos específicos}
Identificar principios, analizar responsabilidad compartida, evaluar controles, determinar riesgos y buenas prácticas.

\section*{5. Marco teórico}
Definiciones: IaaS, PaaS, SaaS; CIA; modelo de responsabilidad compartida; IAM; seguridad de red; seguridad de aplicaciones; normativas.

\section*{6. Desarrollo}
\subsection*{Entorno}
Descripción del entorno utilizado (VM, Docker, servicios GCP si aplica).
\subsection*{Arquitectura}
Diagrama y explicación.
\subsection*{Configuración}
Controles aplicados, IAM, firewall, KMS, logging.
\subsection*{Análisis del ejemplo práctico}
Descripción del experimento, comandos, resultados.

\section*{7. Discusión}
Evaluación, comparación teoría/práctica, riesgos identificados y limitaciones.

\section*{8. Conclusiones}
Cumplimiento de objetivos y aportes.

\section*{9. Recomendaciones}
Buenas prácticas y mejoras.

\section*{10. Fuentes bibliográficas}
Lista de referencias.

\end{document}
